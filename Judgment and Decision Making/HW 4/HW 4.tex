\documentclass[12pt]{article}
\usepackage[letterpaper, portrait, margin=1in]{geometry}
\usepackage{amsmath, amsthm, graphicx, tikz, pgfplots} 

\setlength\parindent{0pt}

\usepackage{fancyhdr}
\pagestyle{fancy}
\fancyhf{}
\lhead{Darshan Patel}
\rhead{Judgment and Decision Making}
\renewcommand{\footrulewidth}{0.4pt}
\cfoot{\thepage}

\begin{document}

\begin{center} \textbf{Homework \#4: Value of Information} \end{center}

A company has just received some state of the art electronic equipment from an overseas supplier. The packaging has been damaged during delivery and the company must decide whether to accept the equipment or send it back. \\

If the equipment itself has not been damaged, it could be sold for a profit of \$10,000. However, if the shipment is accepted and it turns out to be damaged, a loss of \$5,000 will be made. Sending the equipment back will lead to no change in the company's profit. \\

After an inspection, the company's engineer estimates that there is a 60\% chance that the equipment has not been damaged. \\

Suppose that the company has the option to conduct a test on the equipment before they have to make their decision and that this test can tell the company with certainty whether the equipment is damaged or not. \\

What is the most the company would be willing to pay to conduct such a test (i.e. what is the expected value of perfect information)? \\
$$ \includegraphics[width = \textwidth]{partA} $$ 
The expected value without information and with information, respectively, is 
$$ \begin{aligned} E[V]_{\text{without}} &= (0.6 \cdot 10k) + (0.4 \cdot \text{-}5k) = \$4k \\
E[V]_{\text{with}} &= (0.6 \cdot 10k) + (0.4 \cdot 0k) = \$6k \end{aligned} $$ 
Therefore the expected value of perfect information is 
$$ VOI = E[V]_{\text{with}} - E[V]_{\text{without}} = \$6,000 - \$4,000 = \$2,000 $$ 

Now suppose that the test is not perfectly reliable. It can also tell the company whether the equipment is damaged or not, but will only give the correct answer 80\% of the time. \\

What is the most the company would be willing to pay to conduct such a test (i.e. what is the expected value of imperfect information)? \\
$$ \includegraphics[width = \textwidth]{partB_1} $$ 

From this we can create a probability table: 
$$ \begin{tabular}{c|c|c|c} 
& Predicted Damaged (PD) & Predicted Not Damaged (PND) & \\ \hline 
Damaged (D) & 0.32 & 0.08 & 0.4 \\ \hline 
Not Damaged (ND) & 0.12 & 0.48 & 0.6 \\ \hline 
& 0.44 & 0.56 & 1 \end{tabular} $$ 

Using Bayes rule, $$ \begin{aligned} 
P(\text{D} | \text{PD}) &= \frac{ P(\text{PD}|\text{D})P(\text{D})}{P(\text{PD})} = \frac{0.8 \cdot 0.4}{0.44} = 0.727 \\ 
P(\text{ND} | \text{PND}) &= \frac{ P(\text{PND}|\text{ND}) P(\text{ND})}{P(\text{PND})} = \frac{0.8 \cdot 0.6}{0.56} = 0.857 \end{aligned} $$ 
\newpage
Given this information, we can construct the decision tree to figure out the expected value of imperfect information. 
$$ \includegraphics[width = \textwidth]{partB_2} $$ 
If the test predicts not damaged and the company accepts the package, the expected value is $$ E[V] = (0.857 \cdot \$10k) + (0.143 \cdot \text{-}\$5k) = \$7,855 $$ 
which is greater than the alternative of $\$0k$, If the test predicts damaged and the company accepts the package, the expected value is $$ E[V] = (0.273 \cdot \$10k) + (0.727 \cdot \text{-}\$5k) = -\$905 $$ which is less than the alternative of $\$0k$. Hence the expected value with information is $$ E[V]_{\text{with}} = (0.56 \cdot \$7.855k) + (0.44 \cdot \$0k) = \$4,398.8 $$ 
Together with the expected value without information, $\$4,000$, the expected value with imperfect information is $$ VOI = E[V]_{\text{with}} - E[V]_{\text{without}} = \$4,398.8 - \$4,000.00 = \$398.8 $$ 


\end{document} 