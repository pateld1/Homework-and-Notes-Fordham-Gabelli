\documentclass[12pt]{article}
\usepackage[letterpaper, portrait, margin=1in]{geometry}
\usepackage{amsmath, physics}
\newcommand{\ques}[1]{\noindent {\bf Question #1: }} 
\renewcommand{\theenumi}{\alph{enumi}}
\newcommand*\conj[1]{\overline{#1}}
\newcommand{\union}{\cup}
\newcommand{\intersect}{\cap}
\setlength\parindent{0pt}

\usepackage{fancyhdr}
\pagestyle{fancy}
\fancyhf{}
\lhead{Darshan Patel}
\rhead{Statistical Theory II}
\renewcommand{\footrulewidth}{0.4pt}
\cfoot{\thepage}

\begin{document}

\begin{center} \textbf{Assignment \#1: Chapter 7 Questions 10, 48, 62, 92, 96} \end{center}

\ques{7.10} The amount of fill dispensed by a bottling machine is normally distributed with $\sigma = 2$ ounces. \begin{enumerate} 
\item If $n=9$ bottles are randomly selected from the output of the machine, what is $P(\abs{\bar{Y} - \mu} \leq 0.3)$? Compare this to the situation where $\sigma = 1$. \\
If $\sigma = 2$, $$ \begin{aligned} P(\abs{\bar{Y} - \mu} \leq 0.3) &= P(-0.3 \leq \bar{Y} - \mu \leq 0.3) \\ &= P\left(-\frac{0.3}{\sigma / \sqrt{n}} \leq \frac{\bar{Y} - \mu}{\sigma / \sqrt{n}} \leq \frac{0.3}{\sigma \sqrt{n}}\right) \\ &= P\left( -\frac{0.3}{2 / \sqrt{9}} \leq Z \leq \frac{0.3}{2 / \sqrt{9}}\right) \\ &= P(-0.45 \leq Z \leq 0.45) \\ &= \texttt{pnorm(0.45) - pnorm(-0.45)} \\ &= 0.347  \end{aligned} $$ 
Similarly, if $\sigma = 1$, $$ \begin{aligned} P(\abs{\bar{Y} - \mu} \leq 0.3) &= P\left( -\frac{0.3}{1 / \sqrt{9}} \leq Z \leq \frac{0.3}{1 / \sqrt{9}}\right) \\ &= P(-0.9 \leq Z \leq 0.9) \\ &= \texttt{pnorm(0.9) - pnorm(-0.9)} \\ &= 0.831  \end{aligned} $$ 
The probability is smaller when $\sigma=2$. 

\item Find $P(\abs{\bar{Y} - \mu} \leq 0.3)$ when $\bar{Y}$ is to be computed using samples of sizes $n = 25$, $n=36$, $n=49$ and $n=64$. \\
When $n= 25$, $$ \begin{aligned} P(\abs{\bar{Y} - \mu} \leq 0.3)  &= P\left( -\frac{0.3}{2 / \sqrt{25}} \leq Z \leq \frac{0.3}{2 / \sqrt{25}}\right) \\ &= P(-0.75 \leq Z \leq 0.75) = \texttt{pnorm(0.75) - pnorm(-0.75)} = 0.546  \end{aligned} $$ 
When $n= 36$, $$ \begin{aligned} P(\abs{\bar{Y} - \mu} \leq 0.3)  &= P\left( -\frac{0.3}{2 / \sqrt{36}} \leq Z \leq \frac{0.3}{2 / \sqrt{36}}\right) \\ &= P(-0.9 \leq Z \leq 0.9) = \texttt{pnorm(0.9) - pnorm(-0.9)} = 0.631  \end{aligned} $$ 
When $n= 49$, $$ \begin{aligned} P(\abs{\bar{Y} - \mu} \leq 0.3)  &= P\left( -\frac{0.3}{2 / \sqrt{49}} \leq Z \leq \frac{0.3}{2 / \sqrt{49}}\right) \\ &= P(-1.05 \leq Z \leq 1.05) = \texttt{pnorm(1.05) - pnorm(-1.05)} = 0.706  \end{aligned} $$ 
When $n= 64$, $$ \begin{aligned} P(\abs{\bar{Y} - \mu} \leq 0.3)  &= P\left( -\frac{0.3}{2 / \sqrt{64}} \leq Z \leq \frac{0.3}{2 / \sqrt{64}}\right) \\ &= P(-1.2 \leq Z \leq 1.2) = \texttt{pnorm(1.2) - pnorm(-1.2)} = 0.769  \end{aligned} $$ 

\item What pattern do you observe among the values for $P(\abs{\bar{Y} - \mu} \leq 0.3)$ that you observed for the various values of $n$? \\
As $n$ increases, the probability of $\abs{\bar{Y} - \mu} \leq 0.3$ becomes larger and larger.

\item How do the respective probabilities obtained in this problem (where $\sigma=2$) compare to those obtained if $\sigma=1$? \\
The probabilities would be larger because of the smaller standard deviation. \\ For example, if $n= 25$, then 
$$ \begin{aligned} P(\abs{\bar{Y} - \mu} \leq 0.3)  &= P\left( -\frac{0.3}{1 / \sqrt{25}} \leq Z \leq \frac{0.3}{1 / \sqrt{25}}\right) \\ &= P(-1.5 \leq Z \leq 1.5) = \texttt{pnorm(1.5) - pnorm(-1.5)} = 0.866  \end{aligned} $$ This probability is larger than if $\sigma = 2$. The same would apply for the other larger $n$ values.

\end{enumerate}


\ques{7.48} An important aspect of a federal economic plan was that consumers would save a substantial portion of the money that they received from an income tax reduction. Suppose that early estimates of the portion of total tax saved, based on a random sampling of $35$ economists, had mean $26\%$ and standard deviation $12\%$. \begin{enumerate}
\item What is the approximate probability that a sample mean estimate, based on a random sample of $n=35$ economists, will lie within $1\%$ of the mean of the population of the estimates of all economists? 
$$ \begin{aligned} P(\abs{\bar{Y} - \mu} \leq 1) &= P(-1 \leq \bar{Y} - \mu \leq 1) \\ &= P\left( -\frac{1}{s/\sqrt{n}} \leq \frac{\bar{Y} - \mu}{s / \sqrt{n}} \leq \frac{1}{s/\sqrt{n}} \right) \\ &= P\left( -\frac{1}{12 / \sqrt{35}} \leq Z \leq \frac{1}{12/\sqrt{35}}\right) \\ &= P(-0.49 \leq Z \leq 0.49) \\ &= \texttt{pnorm(0.49) - pnorm(-0.49)} \\ &= 0.375 \end{aligned} $$ 

\item Is it necessarily true that the mean of the population of estimates of all economists is equal to the percent tax saving that will actually be achieved? \\ 
This is not true because the mean of the estimates is still an estimation and may or may not be a true representation of the real world.

\end{enumerate}


\ques{7.62} The times that a cashier spends processing individual customer's order are independent random variables with mean $2.5$ minutes and standard deviation $2$ minutes. What is the approximate probability that it will take more than $4$ hours to process the orders of $100$ people? 
$$ \begin{aligned} P(Y_1+Y_2 + \dots + Y_{100} > 60 \cdot 4) &= P(n\bar{Y} > 240) \\ &= P\left(\bar{Y} > \frac{240}{n}\right) \\ &= P\left( \frac{\bar{Y} - \mu}{\sigma / \sqrt{n}} > \frac{\frac{240}{n} - \mu}{\sigma / \sqrt{n}}\right) \\ &= P\left(Z > \frac{\frac{240}{100} - 2.5}{2/\sqrt{100}}\right) \\ &= P(Z > -0.5) \\ &= \texttt{1 - pnorm(-0.5)} \\ &= 0.691 \end{aligned} $$ 


\ques{7.92} From each of two normal populations with identical means and with standard deviations of $6.40$ and $7.20$, independent random samples of $64$ observations are drawn. Find the probability that the difference between the means of the samples exceeds $0.6$ in absolute value. 
$$ \begin{aligned} P(\abs{\bar{X} - \bar{Y}} > 0.6) &= 2 \cdot P\left( \frac{\abs{\bar{X}-\bar{Y}} - (\mu_X - \mu_Y)}{\sqrt{\frac{\sigma_X^2 + \sigma_Y^2}{n}}} > \frac{0.6 - (\mu_X-\mu_Y)}{\sqrt{\frac{\sigma_X^2 + \sigma_Y^2}{n}}}\right) \\ &= 2 \cdot P\left(Z > \frac{0.6 - 0}{\sqrt{\frac{6.40^2 + 7.20^2}{64}}}\right)  \\ &= 2 \cdot P(Z > 0.498) \\ &= \texttt{2 * (1 - pnorm(0.498))} \\ &= 0.618 \end{aligned} $$ 


\ques{7.96} Suppose that $Y_1,Y_2,\dots,Y_{40}$ denote a random sample of measurements on the proportion of impurities in iron ore samples. Let each variable $Y_i$ have a probability density function given by $$ f(y) = \begin{cases} 3y^2 &\text{ if } 0 \leq y \leq 1 \\ 0 &\text{ elsewhere } \end{cases} $$ The ore is to rejected by the potential buyer if $\bar{Y}$ exceeds $0.7$. Find $P(\bar{Y} \geq 0.7)$ for the sample of size $40$. \\
This distribution resembles the Beta distribution with $\alpha = 3$ and $\beta = 1$. Thus $\mu = \frac{\alpha}{\alpha+\beta} = \frac{3}{4}$ and $\sigma^2 = \frac{\alpha\beta}{(\alpha+\beta)^2(\alpha+\beta+1)} = \frac{3}{4^2(5)} = \frac{3}{80}$. Then $$ P(\bar{Y} > 0.7) = P\left(\frac{\bar{Y} - \mu}{\sigma / \sqrt{n}} > \frac{0.7 - \frac{3}{4}}{\sqrt{\frac{3}{80}} / \sqrt{40}}\right) = P(Z > -1.632) = \texttt{1 - pnorm(-1.632)} = 0.948 $$ 



\end{document}