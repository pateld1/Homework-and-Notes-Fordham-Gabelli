\documentclass[12pt]{article}
\usepackage[letterpaper, portrait, margin=1in]{geometry}
\usepackage{amsmath, graphicx}
\newcommand{\ques}[1]{\noindent {\bf Question #1: }} 
\renewcommand{\theenumi}{\alph{enumi}}
\newcommand*\conj[1]{\overline{#1}}
\newcommand{\union}{\cup}
\newcommand{\intersect}{\cap}
\setlength\parindent{0pt}

\usepackage{fancyhdr}
\pagestyle{fancy}
\fancyhf{}
\lhead{Darshan Patel}
\rhead{Statistical Theory I}
\renewcommand{\footrulewidth}{0.4pt}
\cfoot{\thepage}

\begin{document}

\begin{center} \textbf{Assignment \#1: Chapter 2 Questions 148, 162, 166, 174, 178} \end{center}

\ques{2.148} 

A bin contains three components from supplier A, four from supplier B and five from supplier C. If four of the components are randomly selected for testing, what is the probability that each supplier would have at least one component tested? \\~\\
From the $12$ components, $4$ are chosen. Therefore, the probability that each supplier would have at least one component tested is the sum of each supplier getting tested, or  
$$ \frac{ \binom{3}{2}\binom{4}{1}\binom{5}{1} + \binom{3}{1}\binom{4}{2}\binom{5}{1} + \binom{3}{1}\binom{4}{1}\binom{5}{2}}{\binom{12}{4}} = \frac{270}{495} = \frac{6}{11} $$ 


\ques{2.162}

Assume that there are nine parking spaces next to one another in a parking lot. Nine cars need to be parked by an attendant. Three of the cars are expensive sports cars, three are large domestic cars and three are imported compacts. Assuming that the attendant parks the cars at random, what is the probability that the three expensive sports cars are parked adjacent to one another? \\~\\
Since there are $9$ parking spaces, there are $9!$ possible arrangements of parked cars. In the $9$ spaces, there are $7$ possible choices the three expensive sports car can be parked adjacent to each other. In each of these $7$ choices, there are $3!$ ways to put the three expensive sports car together and $6!$ ways to put the other cars together. Hence the probability is $$ \frac{7 \cdot 3! \cdot 6!}{9!} = \frac{30240}{362880} = \frac{1}{12} $$ 


\ques{2.166}

Eight tires of different brands are ranked from $1$ to $8$ (best to worse) according to mileage performance. If four of these tires are chosen at random by a customer, find the probability that the best tire among those selected by the customer is actually ranked third among the original eight. \\~\\
Of the $8$ tires, $4$ are chosen at random; this creates $8$ choose $4$ choices. If the best tire is ranked third among the original $8$, that means the other $3$ chosen tires are from the bottom $5$ ranks. Hence the probability that the best tire among those selected by the customer is actually ranked third among the original eight is $$ \frac{\binom{5}{3}}{\binom{8}{4}} = \frac{10}{70} = \frac{1}{7} $$ 

\newpage
\ques{2.174}

Many public schools are implementing a ``no-pass, no-play" rule for athletes. Under this system, a student who fails a course is disqualified from participating in extracurricular activities during the next grading period. Suppose that the probability is $.15$ that an athlete who has not previously been disqualified will be disqualified next term. For athletes who have been previously disqualified, the probability of disqualification next term is $.5$. If $30\%$ of the athletes have been disqualified in previous terms, what is the probability that a randomly selected athlete will be disqualified during the next grading period? \\~\\
Let event $P$ denote an athlete was disqualified previously. Let event $N$ denote an athlete will be disqualified next term. Then $$ \begin{aligned} P(N ~|~ \overline{P}) &= 0.15 \\ P(N ~|~ P) &= 0.5 \\ P(P) &= 0.3 \end{aligned} $$ 
Then by using the law of total probability, 
$$ \begin{aligned} P(N) &= (P(N ~|~ P) \cdot P(P)) + (P(N ~|~ \overline{P}) \cdot P(\overline{P})) \\ &= (0.5 \cdot 0.3) + (0.15 \cdot 0.7) \\ &= 0.255 \end{aligned} $$ 


\ques{2.178}

Suppose that the probability of exposure to the flu during an epidemic is $.6$. Experience has shown that a serum is $80\%$ successful in preventing an inoculated person from acquiring the flue, if exposed to it. A person not inoculated faces a probability of $.90$ of acquiring the flue if exposed to it. Two persons, one inoculated and one not, perform a highly specialized task in a business. Assume that they are not at the same location, are not in contact with the same person, and cannot expose each other to the flu. What is the probability that at least one will get the flu? \\~\\
Let event $E$ denote a person being exposed to the flu and event $F$ denote a person getting the flu. To solve the probability, find the probability that no one gets the flu first. For the non-inoculated employee, the probability of not getting the flu is $$ P(\overline{F}) = P(\overline{F} \text{ and } E) + P(\overline{F} \text{ and } \overline{E}) = (0.1 \cdot 0.6) + (1.0 \cdot 0.4) = 0.46 $$ 
For the inoculated employee, the probability of not getting the flu is $$ P(\overline{F}) = P(\overline{F} \text{ and } E) + P(\overline{F} \text{ and } \overline{E}) = (0.8 \cdot 0.6) + (1.0 \cdot 0.4) = 0.88 $$ 
Therefore the probability none of the two will get the flu, given there is no dependence, is $$ P(\text{inoculated or non-inoculated}) = P(\text{inoculated}) \cdot P(\text{non-inoculated}) = 0.46 \cdot 0.88 = 0.4048 $$ Hence the probability at least one will get the flu is $$ P(\text{at least one gets the flu}) = 1 - P(\text{inoculated or non-inoculated}) = 1 - 0.4048 = 0.5952 $$ 






\end{document}