\documentclass[11pt]{article}
\usepackage[letterpaper, portrait, margin=1in]{geometry}
\usepackage{amsmath, listings, color} 

\definecolor{mygreen}{rgb}{0,0.6,0}
\definecolor{mygray}{rgb}{0.5,0.5,0.5}
\definecolor{mymauve}{rgb}{0.58,0,0.82}

\lstset{ %
  language=Scala,
  backgroundcolor=\color{white},   % choose the background color
  basicstyle=\footnotesize,        % size of fonts used for the code
  breaklines=true,                 % automatic line breaking only at whitespace
  captionpos=b,                    % sets the caption-position to bottom
  commentstyle=\color{mygreen},    % comment style
  escapeinside={\%*}{*)},          % if you want to add LaTeX within your code
  keywordstyle=\color{blue},       % keyword style
  stringstyle=\color{mymauve},     % string literal style
}

\setlength\parindent{0pt}

\usepackage{fancyhdr}
\pagestyle{fancy}
\fancyhf{}
\lhead{Darshan Patel}
\rhead{Big Data Programming}
\renewcommand{\footrulewidth}{0.4pt}
\cfoot{\thepage}

\begin{document}

\begin{center} \textbf{Assignment \#2} \end{center}
Each question and subquestion has a separate Scala code file.
\begin{enumerate} 

\item Implement the factorial function using \textit{to} and \textit{reduceLeft}, without a loop or recursion. 

\begin{lstlisting}
object Question1 extends App {
	println("The factorial of 5 is " + f(5))
	println("The factorial of 8 is " + f(8))
	
	def f (n: Int): Int = if (n < 1) 1 else (n to 1 by -1).reduceLeft(_*_)
}
\end{lstlisting}

\item Write a Scala program to find the prime number from an array of numbers and print them. 
\begin{lstlisting}
object Question2 extends App {
	val arr = Array(3,7,4,8,12,13,5,23)
	println("List of Numbers to Test:")
	arr.foreach(println) 
	println("The prime numbers are:")
	for(i <- arr if isPrime(i)) println(i)
	
	def isPrime(num:Int): Boolean = {
		if(num <= 1) false
		if(num == 2) true 
		List.range(2, num) forall (x => num % x != 0)
	}
}
\end{lstlisting}

\item Write a Scala code which reads a file and reverse the lines (makes the first line as the last one, and so on). Write the reversed file to a new file named ``reversed.txt" at the same location.
\begin{lstlisting}
object Question3 extends App {
	import scala.io.Source
	import java.io._

	reverseLines("alice.txt")

	def reverseLines(f: String){
		println("File is being read.")
		val input = Source.fromFile(f)
		val lines = input.getLines.toArray
		val rev = lines.reverse 
		val writer = new PrintWriter(new File("rev.txt"))
		println("File is being written.")
		rev.foreach(writer.write)
		writer.close()
	}
}

\end{lstlisting} \newpage

\item Write a Scala code which reads a file and prints all words with more than $10$ characters.
\begin{lstlisting}
object Question4 extends App{
	import scala.io.Source
	import java.io._

	reverseFile("alice.txt", "rev.txt")

	def reverseFile(input: String, output: String){
		println("File is being read.")
		val file = Source.fromFile(input)
		val lines = file.getLines.toArray
		val rev = lines.reverse 
		val writer = new PrintWriter(new File(output))
		println("File is being written.")
		rev.foreach(writer.write)
		writer.close()
	}
}

\end{lstlisting}


\item Write a Scala program to implement QuickSort function. Choose an array of your choice and check the result. 
\begin{lstlisting}
object Question5 extends App {
	var x = Array(6,4,8,3,78,46,26,75,13)
	println("Unsorted:")
	x.foreach(println)
	quicksort(x, 0, x.length - 1)
	println("Sorted:")
	x.foreach(println)

	def swap(arr: Array[Int], i : Int, j:Int) {
		val temp = arr(i)
		arr(i) = arr(j)
		arr(j) = temp
	}

	def quicksort(arr: Array[Int], left: Int, right: Int){
		val split = arr((left + right) / 2)
		var i = left
		var j = right

		while(i < j){
			while(arr(i) < split) i += 1
			while(arr(j) > split) j -= 1
			if (i <= j) {
				swap(arr, i,j)
				i += 1
				j -= 1
			}
		}
		if(left < j) quicksort(arr, left, j)
		if(j < right) quicksort(arr, i,right)
	}
}
\end{lstlisting} \newpage

\item In mathematics, the least common multiple (LCM) of two numbers is the smallest positive integer that can be divided by the two numbers without producing a remainder. LCM can be calculated as follows: $$ LCM(a.b) = \frac{a \cdot b}{GCD(a,b)} $$ where $GCD(a,b)$ is the greatest common divisor of $a$ and $b$, i.e., the largest number that divides both of them without leaving a remainder. Write a Scala program to implement a function to calculate $LCM(a,b)$ using Higher Order Functions. 
\begin{lstlisting}
object Question6 extends App{
	
	println("The LCM of 10 and 49 is: " + lcm(10, 40))
	println("The LCM of 65 and 30 is: " + lcm(65, 30))
	println("The LCM of 3 and 5 is: " + lcm(3,5))
	println("The LCM of 6 and 3 is: " + lcm(6, 3))
	println("The LCM of 12 and 48 is: " + lcm(12, 48))

	def gcd(a: Int, b: Int): Int = {
		if(b == 0) a
		else gcd(b, a % b)
	}

	def lcm(a: Int, b: Int): Int = (a * b) / gcd(a, b)
}
\end{lstlisting}
\newpage
\item OOP with Scala \begin{enumerate} 
\item Write a class BankAccount with methods \textit{deposit} and \textit{withdraw}, and read-only property \textit{balance}. Provide customized getter and setter to check the validity of value of \textit{balance}, e,g., \textit{balance} can only initialized with an amount $\geq 0$. Write a main function to test your class. 
\begin{lstlisting}
object Question7a{
	class BankAccount {
		private var _balance = 0.00
		def this(n: Double){
			this()
			if(n >= 0){
				_balance = n
			}
			else{
				println("This is not tangible money. \n Current balance is reset to 0.00.")
			}
		}

		def currentBalance = _balance
		def deposit(d: Double){
			_balance = _balance + d
		}
		def withdraw(w: Double){
			if(w <= _balance){
				_balance = _balance - w
			}
			else{
				println("You don't have this amount of money.")
			}
		}
	}
	def main(args: Array[String]){
		println("Instantiate Darshan's account with $100.")
		var darshan = new BankAccount(100)
		println("Current Balance: $" + darshan.currentBalance)
		println("Add $5.")
		darshan.deposit(5)
		println("Current Balance: $" + darshan.currentBalance)
		println("Withdraw $1000.")
		darshan.withdraw(1000)
		println("Current Balance: $" + darshan.currentBalance)
		println("Withdraw $12.95")
		darshan.withdraw(12.95)
		println("Current Balance: $" + darshan.currentBalance)
	}
}
\end{lstlisting}
\newpage
\item Extend your BankAccount class to a CheckingAccount class that charges $\$1$ for every \textit{deposit} and \textit{withdraw}. Write a main function to test your CheckingAccount class. 
\begin{lstlisting}
object Question7b{
	class BankAccount {
		private var _balance = 0.00
		def this(n: Double){
			this()
			if(n >= 0){
				_balance = n
			}
			else{
				println("This is not tangible money. \n Current balance is reset to 0.00.")
			}
		}

		def currentBalance = _balance
		def deposit(d: Double){
			_balance = _balance + d
		}
		def withdraw(w: Double){
			if(w <= _balance){
				_balance = _balance - w
			}
			else{
				println("You don't have this amount of money.")
			}
		}
	}
	class CheckingAccount(init: Double) extends BankAccount(init) {
		override def deposit(d: Double){
			super.deposit(d-1)
		}
		override def withdraw(w: Double){
			super.withdraw(w+1)
		}
	}
	def main(args: Array[String]){
		println("Instantiate Darshan's account with $5.")
		var darshan = new CheckingAccount(5)
		println("Current Balance: $" + darshan.currentBalance)
		println("Add $5.")
		darshan.deposit(5)
		println("Current Balance: $" + darshan.currentBalance)
		println("withdraw $1000.")
		darshan.withdraw(1000)
		println("Current Balance: $" + darshan.currentBalance)
		println("Withdraw $2.95.")
		darshan.withdraw(2.95)
		println("Current Balance: $" + darshan.currentBalance)
		println("Add $8.50.")
		darshan.deposit(8.50)
		println("Current Balance: $" + darshan.currentBalance)
	}
}
\end{lstlisting}

\end{enumerate}

\end{enumerate} 

\end{document}